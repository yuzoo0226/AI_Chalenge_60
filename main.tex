%%%%%%%%%%%%%%%%%%%%%%%%%%%%%%%%%%%%%%%%%%%%%%%%%%%%%%%%%%%%%%%%%%%%%%%%%%%%
\documentclass[a4j]{jarticle}

\usepackage{jsaisig}
% \usepackage{graphicx}
\usepackage[dvipdfmx]{graphicx}

%%%%%%%%%%%%%%%%%%%%%%%%%%%%%%%%%%%%%%%%%%%%%%%%%%%%%%%%%%%%%%%%%%%%%%%%%%%

\begin{document}

% 和文タイトル
\title{Find My Matesに向けた解法の提案と実機への実装}

% 英文タイトル
\etitle{JSAI SIGs Conference Paper Format Sample}

% 著者名:
%	・各著者を\quad(全角空白)区切りで列挙
% 	・著者名の直後に\afil{所属番号}を追加→所属番号を上付で出力(\textsuperscript{所属番号}と同じ)
% 	 複数機関へ所属している場合は番号をカンマ区切りで列挙(下記著者2参照)
%  ・Corresponding Authorについては所属の後に\thanksを続け,連絡先を記入
%	・英文著者はカンマ区切りで列挙

\author{矢野 優雅\afil{1}%
	\thanks{連絡先:九州工業大学大学院生命体工学研究科人間知能システム工学専攻 \newline%
		      〒808-0135 福岡県北九州市若松区ひびきの2-4 \newline%
		      E-mail: yano.yuuga158@mail.kyutech.jp}\quad%
	福田 有輝也\afil{1}
	小野 智寛\afil{1}
	田向 権\afil{1,2}\\
	Yuga Yano\afil{1}, \quad \quad Yukiya Fukuda\afil{1}, \quad \quad Tomohiro Ono\afil{1}, and \quad \quad Hakaru Tamukoh\afil{1,2}}

% 所属
\affiliation{%
	\afil{1} 九州工業大学大学院生命体工学研究科\\
	\afil{1} Graduate School of Life Science and Systems Engineering, Kyushu Institute of Technology\\
	\afil{2} ニューロモルフィックAIハードウェア研究センター\\
	\afil{2} Research Center for Neuromorphic AI Hardware}

\abstract{
Abstract (English) comes here.......................................................
}

\maketitle
\thispagestyle{empty}

%%%%%%%%%%%%%%%%%%%%%%%%%%%%%%%%%%%%%%

\section{序論}
近年,少子高齢化の影響によってホームサービスロボットへの注目が集まっている.
また,ホームサービスロボットの発展を目的として,RoboCup@Homeが開催されている.
RoboCup@Homeは,実際の家庭環境を模した部屋でロボットを動作させ,様々なタスクに挑戦し得点を競う大会である.

\subsection{RoboCup@Home}
RoboCup@Homeは,サービスロボット技術の発展を目的に開催されている競技会である.

% The RoboCup@Home league aims to develop service and assistive robot technology with high relevance for future personal domestic applications. It is the largest international annual competition for autonomous service robots and is part of the RoboCup initiative. A set of benchmark tests is used to evaluate the robots’ abilities and performance in a realistic non-standardized home environment setting. Focus lies on the following domains but is not limited to: Human-Robot-Interaction and Cooperation, Navigation and Mapping in dynamic environments, Computer Vision and Object Recognition under natural light conditions, Object Manipulation, Adaptive Behaviors, Behavior Integration, Ambient Intelligence, Standardization and System Integration. It is colocated with the RoboCup symposium.

\subsection{Human Support Robot}
Human Support Robot(HSR)はトヨタ自動車が開発したロボットで,

\subsection{Find My Mates}

%%%%%%%%%%%%%%%%%%%%%%%%%%%%%%%%%%%%%%

\section{関連研究}


\section{提案手法}

\subsection{音声認識}


\subsection{サブセクション}
□□□□□□□□□□□□□□□□□□□□
□□□□□□□□□□□□□□□□□□□□
%
\begin{figure}[ht]
  \centering
  \includegraphics[width=2cm]{images/hsr_front.png}
  \caption{図の挿入例.}
  \label{fig:ex1}
\end{figure}
%
□□□□□□□□□□□□□□□□□□□□
□□□□□□□□□□□□□□□□□□□□
□□□□□□□□□□□□□□□□□□□□
□□□□□□□□□□□□□□□□□□□□
□□□□□□□□□□□□□□□□□□□□
%
\begin{table}[h]
  \centering
  \caption {表の挿入例.}
  \label{table:ex1}
  \begin{tabular}{c|cc}
    \hline
           & $a_1$ & $a_2$ \\ \hline
    $x_1$  &  0.1  &  0.1  \\
    $x_2$  &  0.2  &  0.2  \\ \hline
  \end{tabular}
\end{table}
%
□□□□□□□□□□□□□□□□□□□□
□□□□□□□□□□□□□□□□□□□□
....

図表の参照例:図\ref{fig:ex1},表\ref{fig:ex1}

参考文献の引用例:\cite{Sample1}\cite{Sample2}

%%%%%%%%%%%%%%%%%%%%%%%%%%%%%%%%%%%%%%

\section{むすび}
□□□□□□□□□□□□□□□□□□□□
□□□□□□□□□□□□□□□□□□□□
....

%%%%%%%%%%%%%%%%%%%%%%%%%%%%%%%%%%%%%%

\section*{謝辞}

□□□□□□□□□□□□□□□□□□□□
□□□□□□□□□□□□□□□□□□□□
....

%%%%%%%%%%%%%%%%%%%%%%%%%%%%%%%%%%%%%%

\begin{thebibliography}{99}
%\small

\bibitem{vosk}
Author, A., Author, B.:
JSAI SIGs Conference Paper Format Sample,
{\it International Journal of Examples}, Vol.~19, No.~4, pp.~1--2 (2007)

\bibitem{yolo}
第一著者, 第二著者:
人工知能学会研究会原稿フォーマットサンプル,
{\it International Journal of Examples}, Vol.~19, No.~4, pp.~1--2 (2007)

\end{thebibliography}

\end{document}
